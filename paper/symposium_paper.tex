% iaus2esa.tex -- sample pages for Proceedings IAU Symposium document class
% (based on v1.0 cca2esam.tex)
% v1.04 released 17 May 2004 by TechBooks
%% small changes and additions made by KAvdH/IAU 4 June 2004
% Copyright (2004) International Astronomical Union

\NeedsTeXFormat{LaTeX2e}

\documentclass{iau} 
\usepackage{graphicx}
\usepackage{physics}


% Math operators:
\DeclareMathOperator{\ra}{\mathrm{Ra}}
\DeclareMathOperator{\rac}{\mathrm{Ra_c}}
\DeclareMathOperator{\ek}{\mathrm{E}}
\DeclareMathOperator{\pr}{\mathrm{Pr}}
\DeclareMathOperator{\prmag}{\mathrm{Pm}}
\DeclareMathOperator{\di}{\mathrm{Di}}

\title[Dynamo simulations with double layer] %% give here short title %%
{Dipolar stability in spherical simulations: the impact of an inner stable zone}

\author[Bonnie R. Zaire \& Laurene Jouve]   %% give here short author list %%
{Bonnie R. Zaire$^1$
%%  \thanks{Present address: Fluid Mech Inc., 24 The Street, Lagos, Nigeria.},
 \and Laurene Jouve$^2$}

\affiliation{IRAP, Université de Toulouse, CNRS / UMR 5277, CNES, UPS, 14 avenue E. Belin, Toulouse, F-31400 France
 \\[\affilskip]
$^1$email: {\tt bzaire@irap.omp.eu} ;  $^2$email: {\tt ljouve@irap.omp.eu}}

\pubyear{2019}
\volume{xxx}  %% insert here IAU Symposium No.
\setcounter{page}{1}
\jname{Title of your IAU Symposium}
\editors{A.C. Editor, B.D. Editor \& C.E. Editor, eds.}
\begin{document}

\maketitle

\begin{abstract}

\keywords{Dynamo, Stably stratified, Numerical simulation, etc.}
%% add here a maximum of 10 keywords, to be taken form the file <Keywords.txt>
\end{abstract}

\firstsection % if your document starts with a section,
              % remove some space above using this command.
\section{Introduction}


\section{Governing equations}

We perform $3D$ magnetohydrodynamic simulations of a stratified fluid in a spherical shell of outer radius 
$r_\mathrm{o}$ and inner radius $r_\mathrm{i}$. We use the anelastic version of the code MagIC to solve the 
non-dimensional equations that govern convective motions and magnetic field generation.

\begin{equation}
\pdv{\va{u}}{t}  + \va{u} \cdot \grad \va{u} + \frac{2}{\ek}\vu{e}_z\cross\va{u}  = - \grad p^\star + \frac{\ra}{\pr} gs'\vu{e}_r  + \frac{1}{\prmag\ek\bar{\rho}}(\curl \va{B} )\cross\va{B}
+ \va{F}_\nu, 
\end{equation}

\begin{equation}
\div(\bar{\rho} \va{u}) = 0,  
\end{equation}

\begin{equation} 
\bar{\rho}\bar{T}\left(\pdv{s'}{t}  + \va{u} \cdot \grad s'\right) +u_\mathrm{r}\dv{\bar{s}}{r}  = \frac{1}{\pr}  \div(\bar{\kappa} \bar{\rho} \bar{T} \grad s')  
+ \frac{\pr\di}{\ra}Q_\nu + \frac{\pr\di}{\prmag^2\ek\ra}\lambda(\curl\va{B})^2,
\end{equation}

\begin{equation} 
\pdv{\va{B}}{t}   = \curl( \va{u}\cross\va{B} ) -\frac{1}{\prmag}\curl( \lambda \curl\va{B}),
\end{equation}

\begin{equation}
\div \va{B} = 0,  
\end{equation}




Our reference state is represented by an ideal gas nearly adiabatic, given by
\begin{equation} 
\frac{1}{\bar{T}}\pdv{\bar{T}}{r}  =  \epsilon_\mathrm{s} \dv{\bar{s}}{r}  - \di \bar{\alpha}g(r),  
\end{equation}

\begin{equation} 
\frac{1}{\bar{\rho}}\pdv{\bar{\rho}}{r}  =  \epsilon_\mathrm{s} \dv{\bar{s}}{r}  - \frac{\di \bar{\alpha}}{\Gamma}g(r)
\end{equation}


\section{Implications}

\begin{thebibliography}{}

\bibitem[Amari \etal\ (1995)]{Amari_etal95}
{Amari, S., Hoppe, P., Zinner, E., \& Lewis R.S.} 1995,
\textit{Meteoritics}, 30, 490 


\end{thebibliography}
\end{document}

